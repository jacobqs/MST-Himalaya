

% On the first page, you should present
% The area of research (e.g. implementation of information systems)
% The most relevant previous findings in this area
% Your research problem and why this is worthwhile studying
% The objective of the thesis: how far you hope to advance knowledge in the field

% Why of interest?
% What do we know? What is missing?
% Aim

% A survey of the literature (journals, conferences, book chapters) on the areas that are relevant to your research question. One section per area.

% The chapter should conclude with a summary of the previous research results that you want to develop further or challenge. The summary could be presented in a model, a list of issues, etc. Each issue could be a chapter in the presentation of results. They should definitely be discussed in the discussion / conclusion of the thesis. 



% \section{Motivation}

% \section{Goal and objectives}

% \section{Scope}

% \section{Thesis outline}

% \section{Hydrological modeling: Literature review}

\section{Motivation}

The water supply from the Himalayan and Tibetan Plateau is of great importance for 
millions of people \autocite{bookhagenCompleteHimalayanHydrological2010}. The water from this region provides drinking water, 
supports agricultural demands, is used for hydropower generation and other agro-economic 
activities \autocite{menegozPrecipitationSnowCover2013}. The countries in the High Mountain Asia (HMA) 
region have a huge potential for developing hydropower making the shift towards a greener 
economy more feasible. The Bundhi-Gandaki catchment is located in the Gorkha district of Nepal. 
The catchment has a high mean annual precipitation of 1495 mm and a substantial spatial variation in 
elevation \autocite{devkotaClimateChangeAdaptation2017}. For example, the lowest point is at 479 meters above sea 
level (m.a.s.l.) at Arughat hydrological station and the highest at Manaslu mountain 8163 m.a.s.l. The 
combination of high precipitation rate and steep gradients in elevation makes the region of superior 
interest for hydropower. The installed capacity is 1200 megawatt (MW) with an average energy generation 
of 3383 gigawatt-hours (GWh). However, the hydropower potential is dependent on the climatic conditions 
in precipitation, evaporation, temperature and snow/ice in the catchment \autocite{edenhoferRenewableEnergySources2011}. 
Climate change has serious implications fro hydropower production \autocite{dandekhyaGandakiBasinMaintaining2017}. 
Changing rainfall pattern and increased temperatures will affect power generation. The retreat of 
glaciers, expansion of glacial lakes and changes in the seasonality and intensity of rainfall is 
identifies as factors that will affect the power generation in the future \autocite{dandekhyaGandakiBasinMaintaining2017}.
The region is also threatened by Glacial Lake Outburst Floods (GLOFs), that can cause devastating floods 
downstream \autocite{dandekhyaGandakiBasinMaintaining2017}. In addition, the extremely heterogeneous topography of the 
catchment makes it challenging to get accurate measurements of meteorological 
variables \autocite{pellicciottiChallengesUncertaintiesHydrological2012}. The scarcity of data makes the discharge predictions, 
which lies the foundation for hydropower, prone to errors. It is impossible to make observations at 
all high levels, especially for snow. For this reason, satellite observations and reanalysis data sets 
important for decision-making processes. Furthermore, different precipitation patterns on the leeward and 
windward sides of the catchment needs to taken into account in simulations.


\section{Aim and objectives}

\section{Scope}

\section{Thesis outline}